\documentclass{beamer}
\usepackage{attrib}
\usepackage{hyperref}
\usetheme{Copenhagen}

\title{Git Branching and Merging}
\author{Mike DeLaurentis}
\date{May 13, 2013}
\date{}

\begin{document}

\begin{frame}[plain]
\titlepage
\end{frame}

\begin{frame}[plain]
\begin{huge}

\tableofcontents

\end{huge}
\end{frame}

\section{Commits and branches explained}

\begin{frame}[plain]
\begin{huge}
Why branch?
\end{huge}
\end{frame}

\begin{frame}[plain]
\begin{LARGE}
\begin{itemize}
\item Keep unrelated efforts separate from each other until ready
\item Separate experimental or in-progress features from stable code
\item Maintain diverging versions of code
\item Quickly patch a production issue, without including half-done features
\end{itemize}
\end{LARGE}
\end{frame}


\begin{frame}[plain]
\begin{center}
\begin{Large}
 In older source control systems,

\vspace{0.2in}

 (CVS, Subversion),

\vspace{0.2in}

 merging was hard
\end{Large}
\end{center}

\end{frame}

\begin{frame}[plain]
  \begin{center}
    \includegraphics[scale=0.6]{now_i_cant_merge.jpg}

    \begin{scriptsize}
      http://cdn.memegenerator.net/instances/400x/34202554.jpg
    \end{scriptsize}
  \end{center}
\end{frame}


\begin{frame}[plain]
  But in git, merging is much easier, and branches are much more common.
\end{frame}

\begin{frame}[plain]



\begin{Large}
A git commit:

\begin{itemize}
\item Represents the state of the repository at some time
\item Has a pointer to its parent commit(s),
\item and changes to bring repository into current state.
\item Is atomic.
\end{itemize}
\end{Large}
\end{frame}

\begin{frame}
\begin{Large}
A git branch:
\begin{itemize}
\item Is a pointer to a commit.
\item That is all.
\end{itemize}
\end{Large}
\end{frame}

\begin{frame}
\begin{Large}
A merge commit:
\begin{itemize}
\item Has two parents it points to,
\item and the changes necessary to merge those two parents,
\item (if there are conflicts).
\end{itemize}
\end{Large}
\end{frame}

\begin{frame}[plain]
  \begin{Large}
Git likes committing, branching, and merging:
\begin{itemize}
\item Don't be afraid to commit
\item Don't be afraid of branches.
\end{itemize}
  \end{Large}
\end{frame}

\section{``A successful git branching model'', Vincent Driessen}

\begin{frame}[plain]
\tableofcontents[currentsection]
\end{frame}


\begin{frame}[plain]
  \begin{itemize}
  \item Vincent Driessen writes some recommendations on using git branches
  \item We'll summarize it here
  \item Please see \url{http://nvie.com/posts/a-successful-git-branching-model}
  \end{itemize}
  
\end{frame}


\begin{frame}[plain]

  Maintain a pointer to latest production code

  \begin{itemize}
  \item {\tt master} branch
  \item Nothing special about it, just a sensible default name
  \item Head is always production-ready
  \item Never actively committed to
  \item Lives forever
  \end{itemize}

\end{frame}

\begin{frame}[plain]

  Maintain a branch for latest developed features

  \begin{itemize}
  \item {\tt develop} branch
  \item Maybe not production-ready,
  \item but at least shareable
  \item Often branched from, merged back into
  \item Lives forever
  \end{itemize}

\end{frame}

\begin{frame}[plain]

  Make a branch for a feature or small set of features

  \begin{itemize}
  \item ``feature'' branch, named after specific feature
  \item Short-lived
  \item Branch from ``develop''
  \item Merge back into ``develop''
  \item Delete when finished with it
  \end{itemize}

\end{frame}


\begin{frame}[plain]

  Make a little branch to finalize a release

  \begin{itemize}
  \item {\tt release-}x.x branch
  \item Used to finish a release (bump version number, release date)
  \item Branch from master
  \item Merge into develop and master
  \item Delete when finished with it
  \end{itemize}

\end{frame}

\begin{frame}[plain]

  Make hot-fixes on their own branches

  \begin{itemize}
  \item {\tt hotfix-}* branch
  \item Used to address issue found in ``production''
  \item Branch from master
  \item Merge into develop and master
  \item Name after version number or issue number
  \item Delete when finished with it
  \end{itemize}

\end{frame}

\begin{frame}[plain]
  \begin{center}
  (Demo)    
  \end{center}

\end{frame}

\end{document}
